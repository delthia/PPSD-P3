\documentclass[12pt,a4paper]{article}
\usepackage[utf8]{inputenc}
%\usepackage[margin=25mm]{geometry}
\setlength{\headheight}{18.0pt}


\usepackage[english,spanish]{babel}
%\usepackage{sectsty}

\usepackage[T1]{fontenc}
%\usepackage{titling}
\usepackage{float}
\usepackage{fancyhdr}
\usepackage{amsmath, amsthm, amsfonts}
\usepackage{graphicx}   % Para \graphicspath
\usepackage{float}      % Para float H
\usepackage{listings}
\usepackage{graphicx}
\usepackage{verbatim}
\usepackage{minted}
\usepackage{xcolor}
%\usepackage{tcolorbox}
%\usepackage{svg}
%\usepackage{placeins}
%\usepackage{fancyvrb}

\definecolor{bgGray}{rgb}{0.88,0.88,0.88}

% Opciones del documento
\usepackage[a4paper, top=3cm, bottom=2cm, left=3cm, right=3cm, marginparwidth=1.75cm]{geometry}
%\usepackage[colorinlistoftodos]{todonotes}
\usepackage[hidelinks]{hyperref}
%\usepackage[style=apa, sorting=nyt]{biblatex}
%\addlibresource{bibliografia.bib}
\usepackage[labelfont=bf,font=rm]{caption} % Formato de las captions

% Bibliografía
%\defbibenvironment{bibliography}
%{\enumerate{}
%{\setlength{\leftmargin}{\bibhang}%
%\setlength{\itemindent}{-\leftmargin}%
%\setlength{\itemsep}{\bibitemsep}%
%\setlength{\parsep}{\bibparsep}}}
%{\endenumerate}
%{\item}

\graphicspath{ {img} }

% METADATOS
\title{
    \begin{figure}[H]
        \begin{center}
        \includegraphics[scale=0.3]{UDC.png}       % Logotipo en color
        %\includegraphics[scale=0.35]{UDC-BN.png}    % Logotipo en blanco y negro
        \end{center}
        \label{fig:udc}
    \end{figure}

    \textsf{UNIVERSIDADE DA CORUÑA} \\
    \textsf{\Large Facultade de Informática} \\

    \vfill
    \textbf{PPSD, Práctica 3: \\ Protección de datos II}
}

\author{
    Losada Sánchez, Alicia \\
    \texttt{\href{mailto:alicia.losada.sanchez@udc.es}{alicia.losada.sanchez@udc.es}}
    \\ \\
    Muñiz Rodríguez, Nicolás \\
    \texttt{\href{mailto:nicolas.muniz@udc.es}{nicolas.muniz@udc.es}}
    \\ \\
    Rivas Moar, Iago \\
    \texttt{\href{mailto:iago.rivas@udc.es}{iago.rivas@udc.es}}
    \\ \\
    \vspace{5cm}
}
\date{\textsf\today}


\usepackage{titling}
% espacio de antes
\setlength{\droptitle}{-2cm}
% espacio DESPUÉS del \maketitle
\posttitle{\vspace{0.5cm}}


\pagestyle{fancy}
\fancyhead[L]{\textbf{PPSD - Práctica 3}}
\fancyhead[C]{\large\textbf{Protección de datos II}}
\fancyhead[R]{\textbf{GCED}}
\fancyfoot{}


\begin{document}
\maketitle

%\selectlanguage{english}
%\begin{abstract}
%    On the other hand, we denounce with righteous indignation and dislike men who are so beguiled and demoralized by the charms of pleasure of the moment, so blinded by desire, that they cannot foresee the pain and trouble that are bound to ensue; and equal blame belongs to those who fail in their duty through weakness of will, which is the same as saying through shrinking from toil and pain. These cases are perfectly simple and easy to distinguish. In a free hour, when our power of choice is untrammelled and when nothing prevents our being able to do what we like best, every pleasure is to be welcomed and every pain avoided. But in certain circumstances and owing to the claims of duty or the obligations of business it will frequently occur that pleasures have to be repudiated and annoyances accepted.
%\end{abstract}
\thispagestyle{empty}

\newpage
\selectlanguage{spanish}

\tableofcontents
\thispagestyle{fancy}

\newpage
\fancyfoot[C]{\thepage}

\section{Criptografía moderna}
\subsection{Ejercicio 1}
\graphicspath{ {img/1} }

\begin{figure}[h]
    \includegraphics[width=15cm]{ClavesRSA-01.png}
    \caption{Generación del perfil del par de claves RSA}
\end{figure}

\begin{figure}[h]
    \includegraphics[width=15cm]{ClavesRSA-02.png}
    \caption{Parámetros públicos de la clave (n, e)}
\end{figure}

\begin{figure}[h]
    \includegraphics[width=15cm]{ClavesRSA-03.png}
    \caption{Pantalla del PIN de usuario}
\end{figure}

\begin{figure}[h]
    \includegraphics[width=15cm]{ClavesRSA-04.png}
    \caption{Texto de ejemplo para encriptar}
\end{figure}

\begin{figure}[h]
    \includegraphics[width=15cm]{ClavesRSA-05.png}
    \caption{Generación del perfil del par de claves RSA}
\end{figure}

\begin{figure}[h]
    \includegraphics[width=15cm]{ClavesRSA-06.png}
    \caption{Generación del perfil del par de claves RSA}
\end{figure}

\begin{figure}[h]
    \includegraphics[width=15cm]{ClavesRSA-07.png}
    \caption{Generación del perfil del par de claves RSA}
\end{figure}

\begin{figure}[h]
    \includegraphics[width=15cm]{EncriptadoRSA-1}
    \caption{Generación del perfil del par de claves RSA}
\end{figure}

\begin{figure}[h]
    \includegraphics[width=15cm]{EncriptadoRSA-2}
    \caption{Generación del perfil del par de claves RSA}
\end{figure}

\begin{figure}[h]
    \includegraphics[width=15cm]{EncriptadoRSA-3}
    \caption{Generación del perfil del par de claves RSA}
\end{figure}

\begin{figure}[h]
    \includegraphics[width=15cm]{DesencriptadoRSA-1}
    \caption{Generación del perfil del par de claves RSA}
\end{figure}

\begin{figure}[h]
    \includegraphics[width=15cm]{DesencriptadoRSA-2}
    \caption{Generación del perfil del par de claves RSA}
\end{figure}


\input{ejercicios/2}
\subsection{Ejercicio 3}
\graphicspath{ {img/3} }

\subsubsection{Generación clave D\&H}

COMPLETAR

\subsubsection{Utilidad clave}

La clave generada mediante el protocolo D\&H (Diffie-Hellman) se usa para establecer un secreto compartido entre dos usuarios, incluso si se comunican a través de un canal inseguro como Internet. Este secreto compartido puede servir, por ejemplo, como clave simétrica para cifrar y descifrar mensajes. 

La funcionalidad de D\&H no es el cifrado de mensajes, sino permitir a las dos partes acordar una clave secreta. 

\subsubsection{Problemas de seguridad}

A pesar de parecer ser un método inteligente para el acuerdo de una clave secreta, si no se toman las medidas de seguridad necesarias, puede llegar a ser un método vulnerable.  

Consideramos que el principal problema con el que nos podemos encontrar es que, como D\&H no autentica las partes participantes, un atacante puede interceptar los valores públicos y establecer claves falsas con ambas partes. Así, el atacante podrá leer y modificar los mensajes sin ser detectado. 

Esto se puede solucionar utilizando distintos tipos de métodos de autenticación, como certificado digital o firma digital, para verificar las identidades participantes. 

Por otra parte, si usamos parámetros débiles como un ‘p’ muy pequeño, como es nuestro caso, se hace posible un ataque por fuerza bruta o logaritmo discreto usando computadoras modernas. 

Esto se puede evitar haciendo uso de claves suficientemente grandes, así como valores seguros de ‘p’ y de ‘g’. 


\section{Certificados digitales}
\input{ejercicios/4}
\input{ejercicios/5}

\section{PGP y S/MIME}
\input{ejercicios/6}
\input{ejercicios/7}
\input{ejercicios/8}
\input{ejercicios/9}
\subsection{Ejercicio 10}
\graphicspath{ {img/10} }

En la actualidad el certificado de persona física de la FNMT no permite su uso para cifrar conversaciones de correo electrónico. Como alternativa podemos generar nuestro propio certificado o utilizar uno de alguna CA reconocida. Decidimos generar nuestros propios certificados firmados por una CA también nuestra con \texttt{openssl}, ya que nos parecía la opción más didáctica.

\subsubsection{Generación de una CA con OpenSSL}

Para crear una CA necesitamos dos cosas, la clave con la que la CA firmara las peticiones de certificado y un certificado público para la CA del que los clientes se puedan fiar para validar los certificados que hayamos firmado.

Decidimos utilizar una clave RSA de 4096 bits. Con el siguiente comando generamos la clave privada de la misma y la almacenamos cifrada con DES:

\begin{minted}[
    frame=single,
    framesep=8pt,
    breaklines,
    bgcolor=bgGray
]{bash}
    openssl genrsa -des3 -out ca.key 4096
\end{minted}

Una vez generada la clave privada, necesitamos un certificado con la clave pública de la CA. Un formato común para compartir las claves públicas junto con su identidad es x509. Con esto, el certificado contiene información sobre la CA y la clave pública. La información puede ser variada, y algunos parámetros habituales son los que pide \\\texttt{openssl} de manera interactiva al generar el certificado

\begin{minted}[
    frame=single,
    framesep=8pt,
    breaklines,
    bgcolor=bgGray
]{bash}
    openssl req -new -x509 -days 1826 -key ca.key -out ca.crt
\end{minted}

Este comando genera el certificado en formato x509 para la clave \texttt{ca.key}. Indicamos que queremos que tenga una validez de 1826 días (5 años) y que se almacene en \\\texttt{ca.crt}.

Ahora tenemos una CA que podemos compartir con todos nuestros amigos para que añadan en sus ordenadores (si se fían de nosotros).

\begin{figure}[H]
    \centering
    \includegraphics[width=\textwidth]{openssl-ca-sombra.png}
    \caption{Generación de una CA con \texttt{openssl}}
\end{figure}

\subsubsection{Generación de certificados}

Con la CA en funcionamiento necesitamos generar certificados para nuestros usuarios. Esto tiene dos partes:

\begin{itemize}
    \item Generación de una petición de certificado por parte del usuario
    \item Generación de un certificado firmado por la CA
\end{itemize}

Es decir, igual que con la CA, el usuario genera una clave privada pero, en este caso, en lugar de generar su propio certificado de clave pública, genera con su parte privada una \textit{petición de certificado}. A continuación, la CA genera con esta información el certificado correspondiente. De esta forma, el usuario obtiene un certificado público firmado por la Autoridad Certificadora sin tener que compartir nunca su clave privada.

La clave privada del usuario será igual que la de la CA, RSA de 4096 cifrada con DES para su almacenamiento seguro, por lo que se puede generar con el mismo comando:

\begin{minted}[
    frame=single,
    framesep=8pt,
    breaklines,
    bgcolor=bgGray
]{bash}
    openssl genrsa -des3 -out iago.rivas.key 4096
\end{minted}

A continuación se genera la petición de certificado CSR \textit{Certificate Signing Request} con \texttt{openssl}

\begin{minted}[
    frame=single,
    framesep=8pt,
    breaklines,
    bgcolor=bgGray
]{bash}
    openssl req -new -key iago.rivas.key -out iago.rivas.csr
\end{minted}

Con esta petición, la autoridad puede firmar el certificado y devolvérselo al usuario para que lo comparta como su clave pulica. Esto se puede hacer con el siguiente comando en \texttt{openssl}:

\begin{figure}[H]
    \centering
    \includegraphics[width=\textwidth]{openssl-key-sombra.png}
    \caption{Generación de la petición de certificado con \texttt{openssl}}
\end{figure}

\begin{minted}[
    frame=single,
    framesep=8pt,
    breaklines,
    bgcolor=bgGray
]{bash}
    openssl x509 -req -days 730 -in iago.rivas.csr -CA ca.crt -CAkey ca.key -set_serial 0xBEEF -out iago.rivas.crt
\end{minted}

Todos los parámetros del certificado los definió el usuario al generar la petición, sin embargo hay algunas cosas que decide la CA, concretamente la duración del certificado, en este caso 730 días (2 años), el número de serie y las características del certificado, aunque en este caso no especificamos ninguna en concreto, por lo que el certificado funcionará tanto para el cifrado de correo electrónico como para autenticación.

Para generar el certificado es necesaria la clave privada de la CA, ya que se incluye la firma de la misma para darle validez ante aquellas personas que se fíen de la Autoridad pero no del usuario del certificado.

\subsubsection{S/MIME en Thunderbird}

Para utilizar nuestro certificado S/MIME en Thunderbird tenemos que añadir primero el certificado de la Autoirdad Certificadora y a continuación nuestro certificado. Necesitamos incluír la parte privada, ya que lo usaremos para firmar y descifrar mensajes, por lo que lo exportaremos en el formato PKCS12 también con openssl, utilizando el comando

\begin{minted}[
    frame=single,
    framesep=8pt,
    breaklines,
    bgcolor=bgGray
]{bash}
    openssl pkcs12 -export -in iago.rivas.crt -inkey iago.rivas.key -out iago.rivas.p12
\end{minted}

Ahora necesitaremos buscar en nuestra cuenta los ajustes de encripción de punto a punto en Thunderbird e importar:
\begin{enumerate}
    \item El certificado de la Autoridad Certificadora
    \item Nuestra clave privada
    \item Las claves públicas de las personas con las que nos queremos comunicar
\end{enumerate}

\begin{figure}[H]
    \centering
    \includegraphics[width=\textwidth]{thunderbird-smime-configuracion.png}
    \caption{Ajustes de S/MIME en Thunderbird}
\end{figure}

Ya podemos firmar y cifrar mensajes del mismo modo que cuando utilizábamos OpenPGP, seleccionando en el mismo desplegable la opción de S/MIME, en lugar de OpenPGP.

\begin{figure}[H]
    \centering
    \begin{subfigure}{.5\textwidth}
        \centering
        \includegraphics[width=\linewidth]{thunderbird-smime-cifrado.png}
        \caption{Envío del mensaje}
    \end{subfigure}%
    \begin{subfigure}{.5\textwidth}
        \centering
        \includegraphics[width=\linewidth]{thunderbird-smime-detalles-cifrado.png}
        \caption{Detalles del mensaje}
    \end{subfigure}
    \caption{Envío de un mensaje cifrado con S/MIME en Thunderbird}
\end{figure}

\begin{figure}[H]
    \centering
    \begin{subfigure}{.5\textwidth}
        \centering
        \includegraphics[width=\linewidth]{thunderbird-smime-firmado.png}
        \caption{Envío del mensaje}
    \end{subfigure}%
    \begin{subfigure}{.5\textwidth}
        \centering
        \includegraphics[width=\linewidth]{thunderbird-smime-detalles-firmado.png}
        \caption{Detalles del mensaje}
    \end{subfigure}
    \caption{Envío de un mensaje firmado con S/MIME en Thunderbird}
\end{figure}

\begin{figure}[H]
    \centering
    \begin{subfigure}{.5\textwidth}
        \centering
        \includegraphics[width=\linewidth]{thunderbird-smime-firmado-cifrado.png}
        \caption{Envío del mensaje}
    \end{subfigure}%
    \begin{subfigure}{.5\textwidth}
        \centering
        \includegraphics[width=\linewidth]{thunderbird-smime-detalles-firmado-cifrado.png}
        \caption{Detalles del mensaje}
    \end{subfigure}
    \caption{Envío de un mensaje firmado y cifrado con S/MIME en Thunderbird}
\end{figure}

\subsection{Ejercicio 11}

Comprobar la firma criptográfica de un paquete es realmente sencillo una vez que utilizamos \texttt{gpg} en los apartados anteriores. Lo primero es importar la clave pública del desarrollador, que como indica la propia web de \href{https://keepassxc.org/}{KeePassXC}, se puede descargar desde el keyserver \url{openpgp.org} con el comando

\begin{minted}[
    frame=single,
    framesep=8pt,
    breaklines,
    bgcolor=bgGray
]{bash}
    gpg --keyserver keys.openpgp.org --recv-keys BF5A669F2272CF4324C1FDA8CFB4C2166397D0D2
\end{minted}

Una vez descargado, podemos comprobar la firma del archivo con el comando

\begin{minted}[
    frame=single,
    framesep=8pt,
    breaklines,
    bgcolor=bgGray
]{bash}
    gpg --verify gpg --verify KeePassXC-2.7.10-x86_64.AppImage.sig
\end{minted}

de la misma forma que hacíamos con los correos electrónicos.

Si podemos asegurar que tenemos la clave pública adecuada una firma criptográfica es una manera mucho más segura de verificar que no se modificó el paquete, ya que el hash es una buena forma de comprobar la integridad que sería muy fácil de evadir en caso de querer servir un paquete comprometido, simplemente tendríamos que mostrar el hash correspondiente al paquete malicioso en la web. Con la firma esto es imposible si ya tenemos descargada la clave pública del desarrollador.

Tanto es así que los gestores de paquetes en GNU/Linux utilizan el método de la firma para verificar la integridad y la seguridad de los paquetes, ya que en esta situación, los paquetes se pueden descargar de diversos repositorios, llamados espejos que podrían gestionar diferentes personas o entidades. Así, todos los paquetes están firmados por el mantenedor de la distribución que estemos utilizando, asegurando que no se modificaron los paquetes independientemente del espejo que utilicemos, siempre que se valide la firma con la clave pública del mantenedor.

\section{Privacidad}
\subsection{Ejercicio 12}
\graphicspath{ {img/12} }

Los sitios web seleccionados fueron:
\begin{itemize}
    \item \url{www.wikipedia.org}
    \item \url{https://elpais.com/}
\end{itemize}

\subsubsection{Cookies almacenadas}

\begin{figure}[H]
    \includegraphics[width=\textwidth]{cookies_wiki.png}
    \caption{Cookies almacenadas por Wikipedia}
    \label{fig:cookies_wiki}
\end{figure}

El primer sitio web seleccionado es la Wikipedia. Para poder ver cuántas cookies se almacenan y cuáles son debemos inspeccionar la página. En \ref{fig:cookies_wiki} podemos ver cuáles son las cookies que se almacenan en el sitio web. Las 6 cookies almacenadas son: “eswikiwmuser-sessionId”, “GeoIP”, “NetworkProbeLimit”, “WMF-Last-Access" (parece duplicada con distinto valor o ámbito de acceso) y “centralnotice_fundraising”.  

Seleccionamos la cookie “GeoIP” para su análisis. Esta cookie contiene información sobre la localización del usuario (por ejemplo, país y región), y presenta los siguientes parámetros: 

\begin{itemize}
    \item \texttt{Expires (Sesión)}: se elimina al cerrar el navegador, lo que limita su persistencia.
    \item \texttt{HttpOnly (false)}: puede ser accedida mediante scripts JavaScript, lo cual implica un riesgo si hubiera un ataque de tipo XSS.  
    \item \texttt{Secure (false)}: puede transmitirse incluso si la conexión no es segura, aunque Wikipedia usa HTTPS por defecto.
    \item \texttt{SameSite (None)}: permite que esta cookie se envíe en peticiones entre sitios (cross-site), lo cual es potencialmente más vulnerable a ataques CSRF si no se combina con otras medidas.
\end{itemize}

Aunque la cookie no contiene datos especialmente sensibles, la combinación de “Secure=false”, “HttpOnly=false” y “SameSite=None” la hace más vulnerable que otras cookies más restringidas. Sin embargo, dado que Wikipedia utiliza HTTPS obligatorio y no utiliza cookies de sesión crítica para usuarios no registrados, el riesgo real es bajo. 

\begin{figure}[H]   
    \includegraphics[width=\textwidth]{cookies_pais.png}
    \caption{Cookies almacenadas por El Pais}
    \label{fig:cookies_pais}
\end{figure}

El segundo sitio web seleccionado es El País. Tal y como hicimos con el dominio anterior, debemos inspeccionar la página. En \ref{fig:cookies_pais} podemos ver cuáles son las cookies que se almacenan en el sitio web. Se almacenan un total de 20 cookies distintas, todas ellas asociadas al mismo dominio, elpais.com.  

Seleccionamos la cookie "AMCVS_" para su análisis. Esta cookie está relacionada con Adobe Analytics y se utiliza para identificar sesiones de usuario. Presenta los siguientes parámetros: 

\begin{itemize}
    \item \texttt{Expires (Sesión)}: se elimina al cerrar el navegador, lo que limita su persistencia.
    \item \texttt{HttpOnly (false)}: puede ser accedida mediante scripts JavaScript, lo cual implica un riesgo si hubiera un ataque de tipo XSS.  
    \item \texttt{Secure (true)}: solo se transmite a través de conexiones HTTPS, lo cual mejora su seguridad.
    \item \texttt{SameSite (None)}: permite que esta cookie se envíe en peticiones entre sitios (cross-site), lo cual es potencialmente más vulnerable a ataques CSRF si no se combina con otras medidas.
\end{itemize}

Aunque la cookie no contiene información directamente sensible, el hecho de que sea accesible desde JavaScript (HttpOnly=false) y pueda ser enviada entre sitios (SameSite=None) la hace más expuesta a ciertos ataques, especialmente si el sitio web tiene fallos de seguridad no corregidos. No obstante, al estar marcada como Secure, su transmisión está protegida frente a interceptación en redes no cifradas. 

\subsubsection{Cookies persistentes y de sesión}

Para determinar si las cookies almacenadas en ambos dominios web, debemos analizar el campo “Expires/Max-Age”. En función de su persistencia, podemos dividir las cookies en dos grupos: persistentes y de sesión. Las cookies persistentes tienen fecha de expiración futura, lo que significa que se guardan incluso después de cerrar el navegador. Al contrario que las anteriores, las cookies de sesión se eliminan al cerrar el navegador. Se identifican fácilmente, pues en el campo mencionado toman el valor “Sesión”. 

En el caso de la Wikipedia, las cookies persistentes son: “NetworkProbeLimit” (expira el 30 de abril de 2025), “WMF-Last-Access 1" (expira el 29 de julio de 2025), “WMF-Last-Access 2" (expira el 1 de junio de 2025) y “centralnotice_fundraising” (expira el 1 de junio de 2025). Las cookies que se borrarán al cerrar el navegador son: “eswikiwmuser-sessionId” y “GeoIP”. 

En el caso de El País, de entre sus 20 cookies distintas, las que se borrarán al cerrar el navegador son: “AMCVS_”, “arc-check”, “ept2” y “sessionfc”. Las 16 cookies restantes son persistentes, ya que muestran fechas de caducidad. El rango de expiración de las cookies persistentes en elpais.com va desde el 30 de abril de 2025 hasta el 4 de agosto de 2026, lo que indica que algunas cookies están pensadas para mantenerse activas durante más de un año, especialmente aquellas vinculadas a analítica o personalización del usuario. 

\subsubsection{Cookies de terceros}

Como se ve en \ref{fig:cookies_wiki}, en el momento de la visita a Wikipedia, no se almacenaron cookies de terceros. Todas las cookies visibles pertenecen al dominio principal es.wikipedia.org, lo cual es coherente con la política de privacidad de Wikipedia, que no utiliza rastreadores ni servicios publicitarios externos en su página principal. 

Por otra parte, como se ve en \ref{fig:cookies_pais}, en el momento de la visita a El País, se observa que todas las cookies están asociadas al dominio elpais.com, por lo que se consideran cookies de primera parte. No obstante, es posible que se carguen cookies de terceros más adelante, si se acepta el uso de publicidad personalizada o se navega más profundamente por el sitio.

\subsubsection{Cookies eliminadas y navegación privada}

Para eliminar todas las cookies de ambos sitios web, debemos hacer click derecho en cualquiera de las cookies que se ven en \ref{fig:cookies_wiki} y \ref{fig:cookies_pais}, y seleccionamos la opción " Eliminar todo de 'es.wikipedia.org'” en el caso de Wikipedia y “Eliminar todo de 'elpais.com'. Una vez hecho esto, procedimos a acceder a ambas webs en modo navegación privada. 

\begin{figure}[H]   
    \includegraphics[width=\textwidth]{cookies_wiki_npriv.png}
    \caption{Cookies almacenadas por Wikipedia en navegación privada tras eliminación}
    \label{fig:cookies_wiki_npriv}
\end{figure}

Al eliminar todas las cookies del sitio wikipedia.org, se borra toda la información previamente almacenada, incluyendo identificadores de sesión, preferencias de idioma o geolocalización (por ejemplo, cookies como GeoIP o WMF-Last-Access). Sin embargo, la navegación por el sitio no se ve afectada, ya que Wikipedia no depende de cookies para su funcionamiento básico, especialmente si el usuario no ha iniciado sesión. 

Al acceder a Wikipedia en modo de navegación privada, se comprueba, como se muestra en \ref{fig:cookies_wiki_npriv}, que no se han generado cookies persistentes ni otros datos de almacenamiento, como service workers o bases de datos. Esto es coherente con el comportamiento de Wikipedia, que aplica una política de privacidad estricta: no utiliza rastreadores ni cookies de terceros, y solo crea cookies mínimas y temporales necesarias para aspectos técnicos. Además, al cerrar la ventana privada, cualquier cookie temporal generada durante la sesión se elimina automáticamente, dejando el navegador sin rastro de la actividad. 

\begin{figure}[H]   
    \includegraphics[width=\textwidth]{cookies_pais_npriv.png}
    \caption{Cookies almacenadas en El País en navegación privada tras eliminación}
    \label{fig:cookies_pais_npriv}
\end{figure}

Al eliminar todas las cookies del sitio elpais.com, se pierde la información relacionada con la sesión, el consentimiento de cookies y otros datos de navegación. Al volver a visitar la página, se muestra nuevamente el aviso de consentimiento, y el sitio crea nuevas cookies mínimas para gestionar la sesión. La navegación no se ve interrumpida, pero se reinicia el seguimiento y es necesario volver a aceptar (o rechazar) las cookies para personalizar la experiencia. 

Tras acceder en modo de navegación privada, como se ve en \ref{fig:cookies_pais_npriv}, el sitio vuelve a generar varias cookies, como AMCVS_, arc_geo, mbox, s_cc, entre otras. Estas cookies permiten el funcionamiento del sitio y ciertos servicios de analítica, pero todas ellas son temporales: al cerrar la ventana privada, se eliminarán automáticamente, sin quedar rastro en el navegador. Esto confirma que El País, aunque instala cookies al cargar, no mantiene ninguna persistente tras finalizar la sesión privada, cumpliendo el comportamiento esperado del modo incógnito. 

\subsection{Ejercicio 13}
\graphicspath{ {img/13} }

\subsubsection{Proceso aceptar cookies necesarias}

Según la Agencia Española de Protección de Datos, los sitios web deben ofrecer al usuario la posibilidad de aceptar o rechazar las cookies no necesarias de forma clara y sencilla, sin que aceptar las cookies sea más fácil que rechazarlas \cite{AEPD}. Teniendo esto en cuenta, consideramos que los tres sitios web mencionados deberían permitir aceptar solamente las cookies necesarias.

Las políticas de cookies de los sitios web son:
\begin{itemize}
    \item \url{https://www.udc.es/es/pe/cookies/}
    \item \url{https://www.lavozdegalicia.es/docs/politica_de_cookies.htm}
    \item \url{https://www.marca.com/cookies.html}
\end{itemize}

La política de cookies de la Universidad da Coruña (UDC), que también se aplica al catálogo CRUNIA, menciona que se pueden "restringir, bloquear o eliminar" manualmente las cookies no necesarias desde la configuración de tu propio navegador, teniendo en cuenta que no se podrá hacer uso de algunas funcionalidades indicadas en las propias cookies.

\begin{figure}[H]
    \centering
    \includegraphics[width=\textwidth]{cookies_voz.png}
    \caption{Aviso cookies inicio La Voz de Galicia}
    \label{fig:cookies_voz}
\end{figure}

En el sitio web de La Voz de Galicia, al acceder por primera vez, como se ve en la figura \ref{fig:cookies_voz}, se presenta un aviso de cookies que permite al usuario gestionar sus preferencias de manera personalizada. Este aviso ofrece la opción de aceptar todas las cookies o configurar cuáles se desean aceptar, incluyendo la posibilidad de rechazar las cookies no esenciales.

Para aceptar únicamente las cookies necesarias, el usuario debe seleccionar la opción de configuración y desactivar las categorías de cookies no esenciales, como las de análisis estadístico, geolocalización o publicidad comportamental. Este proceso, aunque requiere varios pasos, está diseñado para ser claro y accesible, cumpliendo con las directrices de la Agencia Española de Protección de Datos.

Además, la política de cookies de La Voz de Galicia detalla los tipos de cookies utilizadas y proporciona enlaces a las políticas de privacidad de terceros involucrados, como Google Analytics, Chartbeat o Facebook. Esto permite al usuario tomar decisiones informadas sobre su privacidad y el uso de sus datos personales.

\begin{figure}[H]
    \centering
    \includegraphics[width=\textwidth]{cookies_marca.png}
    \caption{Aviso cookies inicio MARCA}
    \label{fig:cookies_marca}
\end{figure}

\begin{figure}[H]
    \centering
    \includegraphics[width=\textwidth]{panel_cookies_marca.png}
    \caption{Panel configuración cookies MARCA}
    \label{fig:panel_cookies_marca}
\end{figure}

Tal y como sucedía en La Voz de Galicia, en el sitio web de MARCA, al acceder por primera vez, como se ve en la figura \ref{fig:cookies_marca}, se presenta un aviso de cookies que permite al usuario gestionar sus preferencias de manera personalizada. Este aviso ofrece la opción de aceptar todas las cookies o configurar cuáles se desean aceptar, incluyendo la posibilidad de rechazar las cookies no esenciales.

En la Política de Cookies se recalca que se hace uso de dos tipos distintos de cookies: estrictamente necesaria y de configuración. Las cookies estrictamente necesarias son aquellas que permiten la navegación a través del sitio web, garantizan el correcto funcionamiento del mismo y la utilización de las diferentes opciones o servicios que en él existen. Se indica que si se desactivan estas cookies, el usuario no podrá recibir correctamente nuestros contenidos y servicios. Por otra parte, las cookies de configuración son aquéllas que permiten recordar información para que el usuario acceda al servicio con determinadas características que pueden diferenciar su experiencia de la de otros usuarios.

En el sitio web de MARCA, sí se permite aceptar únicamente las cookies estrictamente necesarias, pero el proceso no es directo. Aunque existe la opción de "Rechazar todo", como se ve en la figura \ref{fig:panel_cookies_marca}, que evita la activación de cookies no esenciales, si el usuario quiere configurar su consentimiento de manera más personalizada, debe revisar manualmente varias categorías de finalidades (como localización, análisis estadístico o desarrollo de servicios) y rechazar o aceptar cada una de ellas individualmente.

Por tanto, el proceso para aceptar sólo las cookies necesarias resulta algo engorroso en comparación con la facilidad de aceptar todas las cookies de un solo clic. Aunque cumple con la normativa de protección de datos, requiere más pasos y atención por parte del usuario para proteger su privacidad de forma selectiva.

De las tres webs analizadas, la política de cookies más engorrosa a la hora de aceptar solo las cookies necesarias es la de MARCA, ya que obliga al usuario a configurar manualmente varias categorías en un panel algo extenso. La Voz de Galicia también requiere acceder a la configuración, pero presenta el proceso de forma algo más clara. En cambio, CRUNIA UDC no ofrece un panel propio de gestión en su web, sino que remite a la configuración del navegador, lo que, aunque limita la personalización, simplifica el proceso de aceptación de cookies necesarias.


\subsubsection{Niveles de personalización}

En el caso de CRUNIA UDC, como ya se mencionó en el anterior apartado, no se ofrece un sistema de personalización de cookies desde el propio sitio web, ya que la UDC informa que puedes gestionar las cookies a través de la configuración de tu navegador. Esto significa que, si deseas limitar o bloquear ciertas cookies, deberás hacerlo manualmente desde las opciones de privacidad de tu navegador. Por lo tanto, el nivel de personalización que ofrece la política de cookies de CRUNIA UDC es básico y depende de las configuraciones que realices en tu navegador. No se proporciona una herramienta integrada en el sitio web para gestionar las preferencias de cookies de manera detallada.

La política de cookies de La Voz de Galicia ofrece un sistema de personalización detallado que permite a los usuarios gestionar sus preferencias de cookies de manera específica. Al acceder al sitio web, se presenta un aviso de cookies que proporciona opciones para aceptar todas las cookies, rechazar las no esenciales o configurar las preferencias de manera individual. Este sistema de gestión cumple con las directrices de la Agencia Española de Protección de Datos, facilitando al usuario el control sobre su privacidad \cite{AEPD}.

En la configuración personalizada, los usuarios pueden activar o desactivar diferentes categorías de cookies, como las técnicas, de análisis estadístico, de geolocalización, de registro, de recomendación de contenidos y publicitarias comportamentales. Además, como se mencionó en el anterior apartado, se proporciona información detallada sobre las finalidades de cada tipo de cookie y se incluyen enlaces a las políticas de privacidad de terceros involucrados, como Google Analytics, Chartbeat o Facebook.

El sitio web de MARCA ofrece un nivel de personalización bastante completo respecto al uso de cookies. A través de un panel de configuración específico de la figura \ref{fig:panel_cookies_marca}, el usuario puede gestionar su consentimiento para diferentes finalidades de tratamiento de datos, como el uso de la geolocalización precisa, el desarrollo y mejora de servicios, la elaboración de estadísticas, o la personalización de contenidos y publicidad. Para cada una de estas finalidades, se puede elegir individualmente entre aceptar o rechazar.

Este nivel de detalle permite al usuario controlar qué tipo de información se recopila y con qué objetivos, aunque requiere revisar varias opciones manualmente. Aun así, este sistema respeta las exigencias legales de transparencia y consentimiento informado, y proporciona un control real sobre el tratamiento de datos personales mediante cookies.


\subsubsection{Venta de datos a terceros y su uso}

La Universidade da Coruña (UDC), a través de su política de cookies, utiliza servicios de terceros para analizar el uso de sus sitios web, incluyendo el catálogo CRUNIA. Uno de estos servicios es Google Analytics, proporcionado por Google LLC. Google Analytics recopila datos como las páginas visitadas, el tiempo de permanencia en ellas, los enlaces en los que se hace clic, el tipo de navegador y dispositivo utilizado, así como la dirección IP de manera anonimizada. Esta información permite a la UDC mejorar la funcionalidad y el contenido de sus plataformas digitales. 

Aunque la UDC no vende los datos recogidos mediante cookies, al utilizar Google Analytics se comparte información con Google para elaborar informes sobre la actividad del sitio, proporcionar servicios relacionados con el uso de Internet, mejorar sus propios servicios y desarrollar otros nuevos.

En el sitio web de La Voz de Galicia, se utilizan cookies de terceros para analizar el comportamiento de los usuarios y personalizar tanto los contenidos como la publicidad mostrada. Entre las empresas que reciben información recopilada a través de estas cookies se encuentra Facebook (Meta Platforms). A través del uso de cookies como las de Facebook, se recoge información sobre la navegación del usuario en el sitio, incluyendo las páginas visitadas, los clics realizados o la duración de la visita.

Esta información se utiliza fundamentalmente para personalizar los anuncios que recibe el usuario, de modo que se ajusten a sus intereses basados en su actividad de navegación. También permite medir el rendimiento de las campañas publicitarias y optimizar la entrega de los anuncios en las plataformas de Facebook.

En la política de cookies de MARCA no se menciona que los datos recogidos se vendan directamente a otras empresas, pero sí se indica que se comparten con terceros colaboradores como Utiq y socios publicitarios, siempre bajo el consentimiento del usuario. Estos datos se utilizan para personalizar la publicidad, analizar estadísticas, desarrollar servicios y mejorar la experiencia de navegación. Además, se informa de que terceros como YouTube, propiedad de Google, también recopilan información de los usuarios a través de vídeos incrustados, combinándola con otros datos de perfil para mostrar publicidad dirigida tanto en servicios de Google como en otros sitios web. Así, aunque no hay una venta explícita de datos, sí se permite su uso por parte de terceros para fines comerciales y publicitarios.

\subsection{Ejercicio 14}
\graphicspath{ {img/14} }

%\begin{itemize}
%    \item \href{https://support.mozilla.org/es/kb/Gestionar-la-configuraci%C3%B3n-del-almacenamiento-local-del-sitio?as=u&utm_source=inproduct&redirectslug=permission-store-data&redirectlocale=en-US}{support.mozilla.org}
%    \item \href{https://librewolf.net/docs/faq/#how-do-i-stay-logged-into-specific-websites}{support.librewolf.org}
%    \item \href{https://addons.mozilla.org/en-US/firefox/addon/cookie-autodelete/}{cookie_autodelete.org}
%    \item \href{https://privacybadger.org/}{privacybadger.org}
%\end{itemize}


\subsubsection{Configuración cookies Mozilla Firefox}

El navegador Mozilla Firefox proporciona múltiples opciones relacionadas con la gestión de cookies [support.mozilla.org]. Entre ellos, se destacan: 

\paragraph{Acceder a los ajustes de almacenamiento del sitio }

Esta opción permite ver y gestionar qué datos (cookies y caché) guarda cada sitio web en el navegador. Para ello, el usuario debe acceder al apartado de preferencias dentro del menú de Firefox. Una vez ahí, como se ve en la \ref{fig:opcion1_ej14} en el apartado de privacidad y seguridad se puede ver información acerca de las cookies del sitio. 

\begin{figure}[H]   
    \includegraphics[width=\textwidth]{opcion1_ej14.png}
    \caption{Opción 1 de configuración de cookies, Firefox}
    \label{fig:opcion1_ej14}
\end{figure}


\paragraph{Eliminar almacenamiento del sitio en páginas individuales }

Esta opción borra los datos almacenados (como cookies) de una página web específica sin afectar a otras. Para poder aprovecharla, una vez estamos en la situación de que se muestra en la \ref{fig:opcion1_ej14} debemos seleccionar la opción "Administrar datos". Como se ve en la \ref{fig:opcion2_ej14}, aparecerá una lista de sitios y cuanta información almacena en el equipo del usuario. Ahí se podrá seleccionar el sitio que se desee para eliminar todas las cookies y datos almacenados. 

\begin{figure}[H]   
    \includegraphics[width=\textwidth]{opcion2_ej14.png}
    \caption{Opción 2 de configuración de cookies, Firefox}
    \label{fig:opcion2_ej14}
\end{figure}

\paragraph{Eliminar los datos almacenados de todos los sitios}

Esta opción elimina todas las cookies y datos guardados de todos los sitios web visitados. Para poder aprovecharla, una vez estamos en la situación de que se muestra en la \ref{fig:opcion1_ej14}, debemos seleccionar la opción “Limpiar datos”. Como se ve en la \ref{fig:opcion3_ej14} se nos permite limpiar información de “cookies y datos del sitio” o “contenido de caché", entre otras. 

\begin{figure}[H]   
    \includegraphics[width=\textwidth]{opcion3_ej14.png}
    \caption{Opción 3 de configuración de cookies, Firefox}
    \label{fig:opcion3_ej14}
\end{figure}

\paragraph{Permitir o bloquear a los sitios que almacenen información }

Esta opción da control al usuario para decidir qué sitios pueden guardar cookies y datos en su dispositivo. Para poder aprovecharla, una vez estamos en la situación de que se muestra en la \ref{fig:opcion3_ej14}, debemos seleccionar la opción “Gestionar excepciones”. Como se ve en la \ref{fig:opcion4_ej14}, se nos permite escribir la dirección exacta del sitio a permitir o bloquear. 

\begin{figure}[H]   
    \includegraphics[width=\textwidth]{opcion4_ej14.png}
    \caption{Opción 4 de configuración de cookies, Firefox}
    \label{fig:opcion4_ej14}
\end{figure}


\subsubsection{Configuración cookies Chrome y LibreWolf}

Para poder ver las opciones de gestión de cookies en Google Chrome debemos acceder al menú de configuración y una vez ahí, como se ve en la \ref{fig:cookies_chrome}, podemos acceder a la configuración de "Cookies de terceros". Ahí nos encontramos con tres opciones: 

\paragraph{Permitir cookies de terceros}

Esta opción permite que todos los sitios web usen cookies para seguimiento, personalización o publicidad. 

\paragraph{Bloquear cookies de terceros en modo Incógnito}

Esta opción impide que terceros usen cookies solo cuando se navega en modo incógnito, limitando el rastreo sin afectar la navegación normal. 

\paragraph{Bloquear cookies de terceros}

Esta opción evita completamente que sitios externos a los que se visita usen cookies, aumentando la privacidad pero pudiendo afectar funciones de algunas webs. 

\paragraph{Enviar una solicitud "Do Not Track" con tu tráfico de navegación }

Esta opción solicita a los sitios que no rastreen la actividad del usuario, aunque pueden ignorarlo porque es voluntario. 

\begin{figure}[H]   
    \includegraphics[width=\textwidth]{cookies_chrome_ej14a.png}
    \caption{Configuración de cookies en Chrome}
    \label{fig:cookies_chrome}
\end{figure}

Por otro lado, para acceder a la configuración de cookies de Librewolf, seguimos los pasos anteriores hasta llevar a la \ref{fig:cookies_librewolf}. Como se aprecia, las opciones son aparentemente las mismas que en Mozilla Firefox, explicadas en el apartado anterior. Esto se debe a que Firefox y LibreWolf son navegadores basados en el mismo motor, pero con enfoques diferentes en cuanto a privacidad y control del usuario. Mientras que Firefox ofrece un equilibrio entre personalización, compatibilidad y privacidad, LibreWolf está diseñado específicamente para proteger al máximo la privacidad desde el primer uso. LibreWolf desactiva por defecto toda la telemetría, bloquea rastreadores y cookies de terceros, elimina automáticamente los datos al cerrar el navegador y no incluye integración con servicios de Mozilla. En cambio, Firefox requiere que el usuario configure manualmente muchas de estas opciones para alcanzar el mismo nivel de privacidad. 

\begin{figure}[H]   
    \includegraphics[width=\textwidth]{cookies_librewolf_ej14a.png}
    \caption{Configuración de cookies en LibreWolf}
    \label{fig:cookies_librewolf}
\end{figure}

En \ref{tab:comparativa-cookies} podemos ver una comparación más visual de los 3 buscadores.

\begin{table}[H]
    \centering
    \begin{tabular}{|l|c|c|c|}
    \hline
    \textbf{Característica} & \textbf{Firefox} & \textbf{LibreWolf} & \textbf{Chrome} \\ \hline
    Permitir todas las cookies & Sí (manual) & No (privacidad estricta) & Sí (por defecto) \\ \hline
    Bloquear cookies de terceros & Sí (opción manual) & Sí (activado por defecto) & Sí (opción disponible) \\ \hline
    Eliminar cookies al cerrar & Opcional & Activado por defecto & Opcional \\ \hline
    Telemetría y rastreo & Activado (puede desactivarse) & Desactivado & Activado (puede limitarse) \\ \hline
    Protección de privacidad & Alta (manual) & Muy alta (por defecto) & Media (requiere configurarlo) \\ \hline
    \end{tabular}
    \caption{Comparativa de opciones de configuración de cookies en Firefox, LibreWolf y Chrome.}
    \label{tab:comparativa-cookies}
\end{table}

\subsubsection{Extensiones de navegador para la gestión de cookies}

\paragraph{Cookie AutoDelete}

Cookie AutoDelete es una extensión de navegador diseñada para gestionar automáticamente las cookies y otros datos de sitios web. Su principal función es eliminar las cookies asociadas a una pestaña en cuanto esta se cierra, evitando que los sitios web rastreen al usuario en futuras visitas. Además, permite crear listas blancas (whitelists) para conservar las cookies de sitios de confianza y listas grises (greylists) para eliminar cookies al reiniciar el navegador. La extensión también ofrece opciones para eliminar manualmente cookies y otros datos de almacenamiento, como IndexedDB y LocalStorage, y es compatible con las pestañas de contenedor en Firefox [\url{cookie_autodelete.gal}]

\paragraph{Privacy Badger}

Privacy Badger, desarrollado por la Electronic Frontier Foundation (EFF), es una extensión de navegador que bloquea automáticamente rastreadores invisibles y cookies de terceros sin necesidad de configuración previa. A diferencia de los bloqueadores tradicionales basados en listas predefinidas, Privacy Badger utiliza un algoritmo de aprendizaje automático que detecta y bloquea dominios en función de su comportamiento de rastreo a lo largo de los sitios web visitados. Además, Privacy Badger envía señales como "Do Not Track" y "Global Privacy Control" para solicitar a los sitios web que no rastreen ni vendan la información del usuario. Si un dominio ignora estas señales y sigue rastreando, Privacy Badger lo bloquea automáticamente, fortaleciendo la privacidad de la navegación sin necesidad de intervención manual [\url{privacybadger.gal}]. 

\paragraph{Pruebas}

Para probar ambas extensiones del navegador, debemos instalarlas. En nuestro caso, elegimos Firefox como navegador. En las \ref{fig:addon_cookie_autodelete} y \ref{fig:addon_privacybadger} se ven las extensiones a instalar. Tras la instalación, se debe activar la extensión en el navegador de Cookie Autodelete, como se ve en la \ref{fig:activacion_cookie_autodelete}, mientras que Privacy Badger aparece automáticamente en la barra de extensiones. 

El siguiente paso es buscar una página web que utilice cookies. Lo ideal es hacer la prueba con páginas grandes que suelan tener rastreadores, como medios de comunicación o redes sociales. En nuestro caso, elegimos la página web del periódico La Voz de Galicia.  

Como se ve en \ref{fig:cookies_lavoz}, al entrar en la página nos aparece una pestaña para gestionar las cookies. Para comprobar el funcionamiento de las extensiones, debemos aceptar las cookies del sitio. Inmediatamente, aparecerán notificaciones de ambas extensiones, indicando que se han detectado correctamente los rastreadores y cookies. En \ref{fig:resultado_cookies_autodelete} y \ref{fig:resultado_privacybadger} vemos los resultados. Destacamos que Privacy Badger señala de color verde a los rastreadores que permite, de amarillo a los rastreadores que permite parcialmente, y de rojo a los bloqueados. 

\begin{figure}[H]   
    \includegraphics[width=\textwidth]{addon_cookie_autodelete.png}
    \caption{Add on de cookie Autodelete}
    \label{fig:addon_cookie_autodelete}
\end{figure}

\begin{figure}[H]   
    \includegraphics[width=\textwidth]{addon_privacybadger.png}
    \caption{Add on de Privacy Badger}
    \label{fig:addon_privacybadger}
\end{figure}

\begin{figure}[H]   
    \includegraphics[width=\textwidth]{activacion_cookie_autodelete.png}
    \caption{Activacion Cookie Autodelete}
    \label{fig:activacion_cookie_autodelete}
\end{figure}

\begin{figure}[H]   
    \includegraphics[width=\textwidth]{cookies_lavoz.png}
    \caption{Cookies la Voz de Galicia}
    \label{fig:cookies_lavoz}
\end{figure}

\begin{figure}[H]   
    \includegraphics[width=\textwidth]{resultado_cookies_autodelete.png}
    \caption{Resultado Cookies Autodelete}
    \label{fig:resultado_cookies_autodelete}
\end{figure}

\begin{figure}[H]   
    \includegraphics[width=\textwidth]{resultado_privacybadger.png}
    \caption{Resultado Privacy Badger}
    \label{fig:resultado_privacybadger}
\end{figure}

\subsection{Ejercicio 15}
\graphicspath{ {img/15} }

\paragraph{Apartado a)} Dentro de las opciones disponibles en el plan gratuito de ProtonVPN podemos destacar \textbf{Quick connect}, que nos permite conectarnos a una VPN escogida automáticamente, en vez de seleccionar manualmente el país/servidor al que conectarnos (en este plan solamente tendremos disponibles conexiones a Japón, Holanda, Polonia, Rumanía y Estados Unidos). Además, contamos con algunas opciones extra en el apartado \texttt{Settings}, como por ejemplo:

\begin{tcolorbox}[
    colback=orange!5!white,
    colframe=orange!75!black,
    title=Opciones relevantes en ProtonVPN
]
\begin{itemize}
    \item \textbf{Kill Switch:} Nos desconecta automáticamente si perdemos la conexión a la VPN.
    \item \textbf{Protocol:} Nos permite cambiar el protocolo de conexión. Podemos elegir entre WireGuard o las dos opciones de OpenVPN: TCP o UDP.
    \item \textbf{IPv6:} Permite filtrar el tráfico que utilice IPv6 a través de la VPN, lo cual aumenta la compatibilidad con redes que utilicen este protocolo.
    \item \textbf{Otras opciones:} También contamos con una opción de cambio de plan además de opciones sobre la aplicación en local, como el iniciarla minimizada o seleccionar nuestras conexiones preferidas como prioritarias.
\end{itemize}
\end{tcolorbox}


\paragraph{Apartado b)} Antes de activar ProtonVPN, miramos las direcciones públicas y privadas de nuestra máquina, además de comprobar nuestra calidad de conexión con el CESGA, mediante RedIRIS.

Como podemos ver en la figura \ref{fig:IPs}, en un principio nuestra IP privada es \\\texttt{192.168.1.141} y la pública \texttt{93.156.217.17}.
Además, nuestra conexión con el CESGA es de \SI{10}{ms} de ping, \SI{2.82}{ms} de jitter, \SI{292}{Mbps} de descarga y \SI{210}{Mbps} de subida.

Estos parámetros son normales dada la distancia que tenemos con el CESGA y que nos estamos conectando directamente.

\begin{figure}[H]
    \centering
    \begin{subfigure}{.5\textwidth}
        \centering
        \includegraphics[width=\linewidth]{IP-Privada.png}
        \caption{Dirección IP Privada}
    \end{subfigure}%
    \begin{subfigure}{.5\textwidth}
        \centering
        \includegraphics[width=\linewidth]{IP-Publica.png}
        \caption{Dirección IP Pública}
    \end{subfigure}
    \caption{Direcciones IP sin ProtonVPN}
    \label{fig:IPs}
\end{figure}


\begin{figure}[H]
    \centering
    \includegraphics[width=\linewidth]{CalidadConexion.png}
    \caption{Calidad de la conexión sin ProtonVPN}
    \label{fig:Calidad-Conexión}
\end{figure}


Ahora, activamos ProtonVPN con Quick Connect, por ejemplo. En nuestro caso nos conectó a un servidor de Holanda.

En la figura \ref{fig:IPs-Holanda} podemos ver el resultado. Se observa que han aparecido dos interfaces de red nuevas, una para IPv4 y otra para IPv6, respectivamente. Esto seguramente sea debido a que activamos la opción IPv6 en los ajustes de ProtonVPN.

En cuanto a las direcciones IP, nuestra dirección privada a utilizar será \texttt{10.98.0.40}, mientras que la pública ha cambiado a \texttt{185.177.126.134}.

Si comprobamos nuestra calidad de red como en la figura \ref{fig:Calidad-Conexión-Holanda} vemos que han aumentado el ping y el jitter, mientras que las velocidades de subida y descarga han disminuído, como es de esperar.

\begin{figure}[H]
    \centering
    \begin{subfigure}{.5\textwidth}
        \centering
        \includegraphics[width=\linewidth]{IP-Privada-Holanda.png}
        \caption{Dirección IP Privada}
    \end{subfigure}%
    \begin{subfigure}{.5\textwidth}
        \centering
        \includegraphics[width=\linewidth]{IP-Publica-Holanda.png}
        \caption{Dirección IP Pública}
    \end{subfigure}
    \caption{Direcciones IP desde Holanda}
    \label{fig:IPs-Holanda}
\end{figure}

\begin{figure}[H]
    \centering
    \includegraphics[width=\linewidth]{CalidadConexion-Holanda.png}
    \caption{Calidad de la conexión desde Holanda}
    \label{fig:Calidad-Conexión-Holanda}
\end{figure}


A continuación repetiremos el proceso otras dos veces más (tres en total) cambiando de servidor. La primera con un servidor en Rumanía y la segunda con un servidor en Japón.
Los resultados se muestran desde la figura \ref{fig:IPs-Rumania} a la figura \ref{fig:Calidad-Conexión-Japon}.

\begin{figure}[H]
    \centering
    \begin{subfigure}{.5\textwidth}
        \centering
        \includegraphics[width=\linewidth]{IP-Privada-Rumania.png}
        \caption{Dirección IP Privada}
    \end{subfigure}%
    \begin{subfigure}{.5\textwidth}
        \centering
        \includegraphics[width=\linewidth]{IP-Publica-Rumania.png}
        \caption{Dirección IP Pública}
    \end{subfigure}
    \caption{Direcciones IP desde Rumania}
    \label{fig:IPs-Rumania}
\end{figure}

\begin{figure}[H]
    \centering
    \includegraphics[width=\linewidth]{CalidadConexion-Rumania.png}
    \caption{Calidad de la conexión desde Rumania}
    \label{fig:Calidad-Conexión-Rumania}
\end{figure}


\begin{figure}[H]
    \centering
    \begin{subfigure}{.5\textwidth}
        \centering
        \includegraphics[width=\linewidth]{IP-Privada-Japon.png}
        \caption{Dirección IP Privada}
    \end{subfigure}%
    \begin{subfigure}{.5\textwidth}
        \centering
        \includegraphics[width=\linewidth]{IP-Publica-Japon.png}
        \caption{Dirección IP Pública}
    \end{subfigure}
    \caption{Direcciones IP desde Japon}
    \label{fig:IPs-Japon}
\end{figure}

\begin{figure}[H]
    \centering
    \includegraphics[width=\linewidth]{CalidadConexion-Japon.png}
    \caption{Calidad de la conexión desde Japon}
    \label{fig:Calidad-Conexión-Japon}
\end{figure}

\subsection{Ejercicio 16}
\graphicspath{ {img/16} }


Antes de realizar el proceso con la VPN de la UDC, comprobamos nuestras direcciones IP privada y pública junto con la calidad de la conexión de la misma manera que en el ejercicio anterior. Mostraremos las características en las figuras \ref{fig:IPs-preUDC} y \ref{fig:Calidad-Conexión-preUDC}.

\begin{figure}[H]
    \centering
    \begin{subfigure}{.5\textwidth}
        \centering
        \includegraphics[width=\linewidth]{IP-Privada-UDC.png}
        \caption{Dirección IP Privada}
    \end{subfigure}%
    \begin{subfigure}{.5\textwidth}
        \centering
        \includegraphics[width=\linewidth]{IP-Publica-UDC.png}
        \caption{Dirección IP Pública}
    \end{subfigure}
    \caption{Direcciones IP antes de utilizar la VPN de la UDC}
    \label{fig:IPs-preUDC}
\end{figure}

\begin{figure}[H]
    \centering
    \includegraphics[width=\linewidth]{CalidadConexion-UDC.png}
    \caption{Calidad de la conexión antes de utilizar la VPN de la UDC}
    \label{fig:Calidad-Conexión-preUDC}
\end{figure}


Para utilizar la VPN de la UDC primero debemos instalarla y configurarla en el enlace \href{https://axudatic.udc.gal/pages/viewpage.action?pageId=45813771}{\texttt{VPN-UDC}}, dentro del apartado \texttt{`Instalación e configuración'}.

Con la VPN instalada,

\subsection{Ejercicio 17}
\graphicspath{ {img/17} }

\subsubsection{Circuito TOR}

El funcionamiento de TOR es crear una red con muchas capas, como si de una cebolla se tratara. El tráfico, en lugar de ir directamente del navegador al servidor de la web a la que el usuario se está intentando conectar pasa por una serie de servidores, los \textit{TOR Relays}. Podemos ver información sobre el circuito seleccionado en la barra de direcciones, como se muestra en la imagen:

\begin{figure}[H]
    \centering
    \includegraphics[width=\textwidth]{tor-google-circuit.png}
    \caption{Circuito con TOR}
\end{figure}

El circuito consta de tres nodos y se pueden cambiar, aunque el primer nodo, el nodo "guardia" no cambiará. Como el tráfico siempre pasa por este nodo durante una misma sesión, independientemente del sitio al que nos conectemos, desde el punto de vista de la operadora, siempre estamos accediendo al mismo sitio.

\subsubsection{Contenido en TOR}

Al conectarnos a través de TOR a \url{google.com} sí que vemos una diferencia con respecto a hacerlo con otro navegador, como Firefox, y es que en lugar de mostrarnos la versión del sitio web para españa, nos muestra la versión del Reino Unido, a pesar de que el nodo de salida de esta sesión está en Países Bajos y es desde ahí desde donde se conectaría a Google.

\begin{figure}[H]
    \centering
    \includegraphics[width=\textwidth]{tor-google.png}
    \caption{\url{google.com} a través de TOR}
\end{figure}

\subsubsection{Cabecera HTTP \texttt{Onion-Location}}

La cabecera HTTP \texttt{Onion-Location} le indica a los usuarios del navegador TOR que el sitio al que están accediendo a través de HTTP, tiene una versión en la red de TOR utilizando el protocolo \texttt{onion}, y cual es su dirección ahí. El sitio web de Proton tiene esta cabecera y por eso en la barra de direcciones vemos un botón que indica que la versión \texttt{.onion} del sitio web está disponible.

\begin{figure}[H]
    \centering
    \includegraphics[width=\textwidth]{proton.me-.onion-popup.png}
    \caption{Botón \texttt{.onion} disponible}
\end{figure}

\begin{figure}[H]
    \centering
    \includegraphics[width=\textwidth]{proton.me-onion-header.png}
    \caption{Detalle de la cabecera \texttt{Onion-Location} en \url{proton.me}}
\end{figure}

\subsubsection{Wikipedia en la red TOR}

Para encontrar la dirección \texttt{.onion} de Wikipedia utilizamos el motor de búsqueda Onion Search Engine que utilizamos para realizar una búsqueda tal y como haríamos en Google u otro buscador:

\begin{figure}[H]
    \centering
    \includegraphics[width=\textwidth]{tor-onionengine.png}
    \caption{Motor de búsqueda Onion Search Engine}
\end{figure}

\begin{figure}[H]
    \centering
    \includegraphics[width=\textwidth]{tor-onionengine-results.png}
    \caption{Resultados de la búsqueda en Onion Search Engine}
\end{figure}

Podemos comprobar que la versión de wikipedia en la red onion es idéntica a su versión en HTTP

\begin{figure}[H]
    \centering
    \includegraphics[width=\textwidth]{tor-wikipedia.png}
    \caption{Versión de Wikipedia en la red Onion}
\end{figure}

\end{document}
