\subsection{Ejercicio 8}
\graphicspath{ {img/08} }

A la hora de utilizar el comando \texttt{gpg} existen una serie de opciones realmente versátiles:

\begin{itemize}
    \item{\texttt{--armor}: Codificar la salida como ASCII, útil para el correo electrónico, que en general solo soporta esta codificación}
    \item{\texttt{--output <path>}: Archivo en el que almacenar la salida del comando utilizado}
    \item{\texttt{--recipients}: Añadir un destinatario que podrá descrifrar el mensaje. Se pueden especificar varios utilizando varias veces esta opción}
    \item{\texttt{--sign}/\texttt{-s}: Firmar la entrada}
    \item{\texttt{--encrypt}/\texttt{-e}: Encriptar la entrada}
    \item{\texttt{--local-user}/\texttt{-u}: Útil para indicar la identidad con la que firmar, en caso de tener varias claves privadas}
    \item{Finalmente se indica el archivo a procesar.}
\end{itemize}

Es útil saber que es posible añadir varios destinatarios para un archivo cifrado, lo que permite simplemente cifrar un mensaje una vez para enviarlo a un hilo de correo en el que participan varias personas.

Otra cosa a tener en cuenta es que suele ser recomendable añadirse a uno mismo como destinatario, lo que permite descifrar los correos ya enviados.

Así, enviar un correo firmado se puede hacer simplemente con el comando

\begin{minted}[
    frame=single,
    framesep=8pt,
    breaklines,
    bgcolor=bgGray
]{bash}
    gpg --sign --armor -u iago.rivas@udc.es cuerpo-correo.txt
\end{minted}

Se indica la identidad con la que firmar el mensaje, ya que dispongo de varias claves privadas.

Para cifrar un correo podemos usar la opción \texttt{encrypt} indicando los recipientes del correo. Además, en este caso no es necesario indicar la identidad, ya que al solo cifrar, nuestra clave privada no se utiliza en ningún momento.

\begin{minted}[
    frame=single,
    framesep=8pt,
    breaklines,
    bgcolor=bgGray
]{bash}
    gpg --encrypt --armor --output correo-cif.asc -r iago.rivas@udc.es -r alicia.losada.sanchez@udc.es -r nicolas.muniz@udc.es cuerpo-correo.txt
\end{minted}

Por último, para firmar un correo además de cifrarlo añadiremos la opción \texttt{sign}. En este caso se vuelve a utilizar la clave privada, ya que al firmar queremos demostrar que el correo se originó en nosotros.

\begin{minted}[
    frame=single,
    framesep=8pt,
    breaklines,
    bgcolor=bgGray
]{bash}
    gpg --encrypt --sign --armor --output correo-cif-sig.asc -r iago.rivas@udc.es -r alicia.losada.sanchez@udc.es -r nicolas.muniz@udc.es -u iago.rivas@udc.es cuerpo-correo.txt
\end{minted}

Aunque al enviar correos con PGP desde Thunderbird se incluyen otros archivos como la clave pública y otros metadatos, al enviar un correo con la firma adjunta, o que solo contiene un archivo adjunto que contiene el contenido cifrado del mensaje, Thunderbird utilizará PGP automáticamente y nos mostrará la información sobre la firma.

Para enviar un mensaje firmado podemos adjuntar el archivo con la firma, que el recipiente podrá comprobar manualmente descargando la firma y el contenido de correo. Thunderbird no detecta automáticamente el archivo de firma.

Para mensajes cifrados es aún más sencillo. Basta con pegar mensaje cifrado en el cuerpo del mensaje. Si el destinatario no tiene thunderbird, puede desencriptar el mensaje manualmente con \texttt{gpg}, mientras que si usa Thunderbird, detectará automáticamente que se trata de un mensaje cifrado y lo descifrará. También comprobará la firma si existe.